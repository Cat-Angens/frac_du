\section{Задача без источника с постоянной скоростью}

Для первого приближения принимается упрощение: скорость переноса $u(x,t)$ принимается за постоянную

\begin{equation}
	u(x,t) \equiv u = const,
\end{equation}
также считаем задачу без источников $f(x,t)\equiv0$, тогда система (\label{eq:def_0}) принимает вид

\begin{equation}
	\label{eq:an_0}
	\begin{cases}
		D^\alpha_t c + uc_x = 0\\
		\left.J^{1-\alpha}_tc\right|_{t=0} = \phi(x)\\
		\left.c\right|_{x=0}=\psi(t)
	\end{cases}
\end{equation}

Тогда применяем преобразование Лапласа $L_t(f(t))$ к главному уравнению системы, учитывая свойство
\begin{equation}
	L_t(D^\alpha_tf(t))(s) = s^\alpha f^* - \left. I^{1-\alpha}_tf \right|_{t=0},
\end{equation}
и получаем
\begin{equation}
	s^\alpha c^*(x,s) - \phi(x) + u c^*_x(x,s) = 0,
\end{equation}
%\begin{equation}
%	c^*(x,s) = \frac{\phi(x)}{s^\alpha + u D_x}.
%\end{equation}
\begin{equation}
	c^*(x,s) = - \phi(x) + u c^*_x(x,s) = 0,
\end{equation}