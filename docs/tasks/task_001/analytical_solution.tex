\section{Задача без источника с постоянной скоростью}

Для первого приближения принимается упрощение: скорость переноса $u(x,t)$ принимается за постоянную

\begin{equation}
	u(x,t) \equiv u = const,
\end{equation}
также считаем задачу без источников $f(x,t)\equiv0$, тогда система (\ref{eq:def_0}) принимает вид

\begin{equation}
	\label{eq:an_0}
	\begin{cases}
		D^\alpha_t c + uc_x = 0\\
		\left.J^{1-\alpha}_tc\right|_{t=0} = \phi(x)\\
		\left.c\right|_{x=0}=\psi(t)
	\end{cases}
\end{equation}

Тогда применяем преобразование Лапласа $L_t(f(t))$ к главному уравнению системы, учитывая свойство
\begin{equation}
	L_t(D^\alpha_tf(t))(s) = s^\alpha f^* - \left. I^{1-\alpha}_tf \right|_{t=0},
\end{equation}
и получаем
\begin{equation}
	s^\alpha c^*(x,s) - \phi(x) + u c^*_x(x,s) = 0,
\end{equation}
\begin{equation}
	c^*_x(x,s) + \frac{s^\alpha}{u}c^*(x,s) = \phi(x).
\end{equation}
Общее решение полученного уравнения:
\begin{equation}
	c^*(x,s) = A_0(s)e^{-\frac{s^\alpha}{u}x} + e^{-\frac{s^\alpha}{u}x} \int_0^x e^{\frac{s^\alpha}{u}\xi} \frac{\phi(\xi)}{u} d \xi.
\end{equation}
С учетом граничного условия получаем
\begin{equation}
	c^*(x,s) = \psi^*(s)e^{-\frac{s^\alpha}{u}x} + \int_0^x e^{\frac{s^\alpha}{u}(\xi - x)} \frac{\phi(\xi)}{u} d \xi.
\end{equation}
Далее необходимо провести обратное преобразование Лапласа. Известны свойства
\begin{equation}
	e^{-\frac{s^\alpha}{u}x} = E_1\left(-\frac{x}{u}s^\alpha; 1\right)
\end{equation}
и
\begin{equation}
	L\left\{t^{\delta - 1}e_{\alpha, \beta}^{\mu, \delta}\left(-ct^{-\beta}\right); s\right\}=
	s^{-\delta}E_{1 / \alpha}\left(-cs^\beta; \mu\right),
\end{equation}
отсюда получаем
\begin{equation}
	L^{-1}\left\{e^{-\frac{s^\alpha}{u}x}; t\right\} = t^{-1} e^{1,0}_{1, \alpha}\left(-\frac{x}{u}t^{-\alpha}\right),
\end{equation}
где $e_{\alpha, \beta}^{\mu, \delta}$ --- функция типа Райта, в случае $\alpha = \mu = 1$ совпадающая с функцией Райта
\begin{equation}
	e_{1, \beta}^{1, \delta}(z) = \phi\left(-\beta, \delta, z\right) =
	\sum_{n=0}^\infty\frac{z^n}{n!\Gamma(-\beta n + \delta)}.
\end{equation}
Тогда
\begin{equation}
	L^{-1}\left\{e^{-\frac{s^\alpha}{u}x}; t\right\} = 
	\sum_{n=0}^\infty\left(-\frac{x}{u}\right)^n \frac{t^{-\alpha n - 1}}{n!\Gamma(-\alpha n)}.
\end{equation}

Используя также свойство преобразования Лапласа от свертки функций
\begin{equation}
	L\left\{\left(f * g\right) \left(t\right); s \right\} = L\left\{(f\left(t\right); s\right\} \cdot L\left\{g\left(t\right); s\right\},
\end{equation}
где
\begin{equation}
	\left(f * g\right) \left(t\right) = \int_0^t f\left(\eta\right) g\left(t - \eta\right) d \eta
\end{equation}
--- свертка функций, получаем решение в исходных координатах:
\begin{equation}
	\begin{split}
		c(x,t) & = \int_{0}^{t} \psi(t - \eta) \sum_{n=0}^\infty\left(-\frac{x}{u}\right)^n \frac{\eta^{-\alpha n - 1}}{n!\Gamma(-\alpha n)} d \eta\\
		& + \int_0^x \frac{\phi(\xi)}{u} \sum_{n=0}^\infty\left(-\frac{\xi-x}{u}\right)^n \frac{t^{-\alpha n - 1}}{n!\Gamma(-\alpha n)} d \xi\\
	\end{split}
\end{equation}
